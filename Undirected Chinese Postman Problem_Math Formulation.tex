\documentclass[a4paper,12pt,titlepage]{article}
\usepackage[utf8]{inputenc} 
\usepackage{tikz,pgf}
\usepackage{indentfirst}
\usepackage{amsfonts}
\usepackage[english]{babel}


%Url e Bookmarks of output PDF 
\usepackage{hyperref}
\hypersetup{
	colorlinks=true,
	linkcolor=blue,
	filecolor=magenta,      
	urlcolor=cyan,
%	pdftitle={Document title},
	bookmarks=true,
	%pdfpagemode=FullScreen,
}




\usepackage{rotating}

\usepackage{tabularx}
\usepackage{multirow} 
\usepackage{lscape}
\usepackage{tikz}
%to insert PDF files
\usepackage[final]{pdfpages}
%--Packages--

\usepackage{eurosym}
\usepackage{graphicx} \usepackage{verbatim}
\usepackage{graphics}
\usepackage{tikz,pgf}
\usepackage{indentfirst}
\usepackage{amsfonts}
\usepackage{graphicx}
\usepackage{amsmath}
\usepackage{amsmath,amssymb,amsthm,textcomp}
\usepackage{enumerate}
\usepackage{multicol}
\usepackage{tikz}
\usepackage{geometry}
\usepackage{mathtools}
\usepackage{amsmath}
\usepackage{verbatim}
\usepackage{amsmath,amssymb,mathrsfs}
\usepackage{xcolor}
\usepackage{graphicx,color,listings}
\frenchspacing 
\usepackage{geometry}
\usepackage{rotating}
\usepackage{caption}
\usepackage{xcolor}
\usepackage{listings}
%Cool maths printing
\usepackage{amsmath}
%PseudoCode
\usepackage{algorithm2e}


\begin{document}
	\section*{Undirected Chinese postman Problem}

The CPP is defined on an undirected graph $G=(V,E)$, where $V=\left\lbrace 1,…, n\right\rbrace $ is the set of vertices, $E =\left\lbrace (i,j)
: i, j\in V, i<j, i\neq j\right\rbrace $ is the set of undirected edges. The traversal cost $c_{ij}$ of an edge $(i,j)$ in $E$ is supposed to be non-negative and is also called cost or distance of $(i,j)$ .\\ 
In case of an edge $(i,j)$ in $E$, it is usually assumed that $c_{ij} = c_ {ji}$. \\
It is generally assumed that $G$ is strongly connected that always possible to reach any vertex from any other vertex.
\subsection*{Sets}
- $V$ set including all nodes in the network;\\
- $E$ set including all edges in the network;\\
- $G=(V,E)$ a connected undirected graph;
\subsection*{Parameters}
- $n$ total number of nodes in network;\\
- $c_{ij}$ distance from $i \in V$ to $j \in V$;
\subsection*{Variables}
- $x_{ij}$ decision variable which represents the number of times arc $(i,j)$ is traversed in each cycle using vehicle starting from node $i\in V$ ending at node $j\in V$.
\begin{equation}
min \sum_{(i,j)\in E}c_{ij} \cdot x_{ij}
\end{equation}

\begin{equation}
\sum_{j=1}^{n} x_{ij} - \sum_{j=1}^{n} x_{ji} = 0 \quad \forall i\in V
\end{equation}

\begin{equation}
x_{ij} + x_{ji} \geq 1 \quad \forall (i,j)\in E
\end{equation}

\begin{equation}
x_{ij}\geq0 \quad \text{and integer} \quad \forall (i,j)\in E
\end{equation}
\newpage
The objective function (1) minimizes the total length of route R that is covered by track inspection vehicle. \\
Eq. (2) is flow conservation at each node constraint which guarantees the creation of a tour of the network for the vehicle. \\
Eq. (3) ensures that each arc that exists is covered at least once during each cycle regardless of its direction using the vehicle. \\
Eq. (4) is restriction on the variables.\\
\\
\small{\textbf{Data and model are excerpts from:}\\
	\textit{Yilmaz, Mustafa \& Kayaci Çodur, Merve \& Yılmaz, Hamid. (2017). Chinese postman problem approach for a large-scale conventional rail network in Turkey. Tehnicki Vjesnik. 24. 1471-1477. 10.17559/TV-20151231153445.}}

\end{document}